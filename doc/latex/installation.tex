\-This section covers how to install \-Force\-Balance. \-Currently only \-Linux is supported, though installation on other \-Unix-\/based systems (e.\-g. \-Mac \-O\-S) should also be straightforward.

\-Importantly, note that {\itshape \-Force\-Balance does not contain a simulation engine\/}. \-Instead it interfaces with simulation software like \-G\-R\-O\-M\-A\-C\-S, \-T\-I\-N\-K\-E\-R, \-A\-M\-B\-E\-R or \-Open\-M\-M; reference data is obtained from experimental measurements (consult the literature) or from quantum chemistry software (for example, \-N\-W\-Chem or \-Q-\/\-Chem).

\-Several interfaces to existing software packages are provided. \-However, if you use \-Force\-Balance for a research project, you should be prepared to write some simple \-Python code to interface with a software package of your choice. \-If you choose to do so, please contact me as \-I would be happy to include your contribution in the main distribution.\hypertarget{installation_installing_forcebalance}{}\subsection{\-Installing Force\-Balance}\label{installation_installing_forcebalance}
\-Force\-Balance is packaged as a \-Python module. \-Here are the installation instructions.

\-A quick preface\-: \-Installing software can be a real pain. \-I tried to make \-Force\-Balance easy to install by providing clear instructions and minimizing the number of dependencies; however, complications and challenges during installation happen all the time. \-If you are running into installation problems or having trouble resolving a dependency, please contact me.\hypertarget{installation_installing_forcebalance_prereq}{}\subsubsection{\-Prerequisites}\label{installation_installing_forcebalance_prereq}
\-Force\-Balance requires the following software packages\-:

\begin{DoxyItemize}
\item \href{http://www.python.org/}{\tt \-Python} version 2.\-7 \item \href{http://numpy.scipy.org/}{\tt \-Num\-Py} version 1.\-5 \item \href{http://www.scipy.org/}{\tt \-Sci\-Py} version 0.\-9\end{DoxyItemize}
\-The following packages are required for certain functionality\-:

\begin{DoxyItemize}
\item \href{http://lxml.de/}{\tt lxml} version 2.\-3.\-4 -\/ \-Python interface to libxml2 for parsing \-Open\-M\-M force field files \item \href{http://nd.edu/~ccl/software/}{\tt cctools} version 3.\-4.\-1 -\/ \-Cooperative \-Computing \-Tools from \-Notre \-Dame for distributed computing\end{DoxyItemize}
\-The following packages are used for documentation\-:

\begin{DoxyItemize}
\item \href{http://www.stack.nl/~dimitri/doxygen/}{\tt \-Doxygen} version 1.\-7.\-6.\-1 \item \href{http://code.foosel.org/doxypy}{\tt \-Doxypy} plugin for \-Doxygen \item \-La\-Te\-X software such as \href{http://www.tug.org/texlive/}{\tt \-Te\-X\-Live}\end{DoxyItemize}
\hypertarget{installation_installing_forcebalance_install}{}\subsubsection{\-Installing}\label{installation_installing_forcebalance_install}
\-To install the package, first extract the tarball that you downloaded from the webpage using the command\-:

\begin{DoxyVerb}tar xvzf ForceBalance-[version].tar.gz \end{DoxyVerb}


\-Alternatively, download the newest \-Subversion revision from the \-Sim\-T\-K website\-:

\begin{DoxyVerb}svn checkout https://simtk.org/svn/forcebalance \end{DoxyVerb}


\-Upon extracting the distribution you will see this directory structure\-:

\begin{DoxyVerb}
<root>
  +- bin
  |   |- <Executable scripts>
  +- src
  |   |- <ForceBalance source files>
  +- ext
  |   |- <Extensions; self-contained software packages that are used by ForceBalance>
  +- studies
  |   +- <ForceBalance example jobs>
  +- doc
  |   +- callgraph
  |   |   |- <Stuff for making a call graph>
  |   +- Images
  |   |   |- <Images for the website and PDF manual>
  |   |- make-all-documentation.sh (Create the documentation)
  |   |- <Below are documentation chapters in Doxygen format>
  |   |- introduction.txt
  |   |- installation.txt
  |   |- usage.txt
  |   |- tutorial.txt
  |   |- glossary.txt
  |   |- <The above files are concatenated into mainpage.py>
  |   |- make-all-documentation.sh (Command for making all documentation)
  |   |- make-option-index.py (Create the option index documentation chapter)
  |   |- header.tex (Customize the LaTex documentation)
  |   |- add-tabs.py (Adds more navigation tabs to the webpage)
  |   |- DoxygenLayout.xml (Removes a navigation tab from the webpage)
  |   |- doxygen.cfg (Main configuration file for Doxygen)
  |   |- ForceBalance-Manual.pdf (PDF manual, but the one on the SimTK website is probably newer)
  |- PKG-INFO (Auto-generated package information)
  |- README.txt (Points to the SimTK website)
  |- setup.py (Python script for installation) \end{DoxyVerb}


\-To install the code into your default \-Python location, run this (you might need to be root)\-:

\begin{DoxyVerb}python setup.py install \end{DoxyVerb}


\-You might not have root permissions, or you may want to install the package somewhere other than the default location. \-You can install to a custom location (for example, to /home/leeping/local) by running\-:

\begin{DoxyVerb}python setup.py install --prefix=/home/leeping/local \end{DoxyVerb}


\-Assuming your \-Python version is 2.\-7, the executable scripts will be placed into {\ttfamily /home/leeping/local/bin} and the module will be placed into {\ttfamily /home/leeping/local/lib/python2.7/site-\/packages/forcebalance}.

\-Note that \-Python does not always recognize installed modules in custom locations. \-Any one of the three below options will work for adding custom locations to the \-Python search path for installed modules\-:

\begin{DoxyVerb}ln -s /home/leeping/local /home/leeping/.local \end{DoxyVerb}
 \begin{DoxyVerb}export PYTHONUSERBASE=/home/leeping/local \end{DoxyVerb}
 \begin{DoxyVerb}export PYTHONPATH=$PYTHONPATH:/home/leeping/local/lib/python2.7 \end{DoxyVerb}


\-As with the installation of any software, there are potential issues with dependencies (for example, scipy and lxml.) \-One way to resolve dependencies is to use the \-Enthought \-Python \-Distribution (\-E\-P\-D), which contains all of the required packages and is free for academic users. \-Install \-E\-P\-D from the \href{http://www.enthought.com/products/epd.php}{\tt \-Enthought website}. \-Configure your environment by running both of the commands below (assuming \-Enthought was installed to {\ttfamily  /home/leeping/opt/epd-\/7.3.\-2 }, with the python executable in the {\ttfamily bin} subdirectory)\-:

\begin{DoxyVerb}
export PATH=/home/leeping/opt/epd-7.3.2/bin:$PATH
export PYTHONUSERBASE=/home/leeping/opt/epd-7.3.2 \end{DoxyVerb}


\-Once you have done this, the \-Numpy, \-Scipy and lxml dependency issues should be resolved and \-Force\-Balance will run without any problems.

\-Here are a list of installation notes (not required if you install \-Force\-Balance into the \-Enthought \-Python \-Distribution). \-These notes assume that \-Python and other packages are installed into \$\-H\-O\-M\-E/local.

\begin{DoxyItemize}
\item \-The installation of \-Numpy, \-Scipy and lxml may be facilitated by installing the {\ttfamily pip} package -\/ simply run a command like {\ttfamily pip install numpy}. \item \-Scipy requires a \-B\-L\-A\-S (\-Basic \-Linear \-Algebra \-Subroutines) library to be installed. \-On certain \-Linux distributions such as \-Ubuntu, the \-B\-L\-A\-S libraries and headers can be found on the repository (run {\ttfamily sudo apt-\/get install libblas-\/dev}). \-Also, \-B\-L\-A\-S is provided by libraries such as \-A\-T\-L\-A\-S (\-Automatically \-Tuned \-Linear \-Algebra \-Software) or the \-Intel \-M\-K\-L (\-Math \-Kernel \-Library) for \-Intel processors. \-To compile \-Scipy with \-Intel's \-M\-K\-L, follow the guide on \href{http://software.intel.com/en-us/articles/numpy-scipy-with-mkl}{\tt \-Intel's website}. \-To use \-A\-T\-L\-A\-S, install the package from the \href{http://math-atlas.sourceforge.net}{\tt \-A\-T\-L\-A\-S website } and set the \-A\-T\-L\-A\-S environment variable (for example, {\ttfamily export \-A\-T\-L\-A\-S=\$\-H\-O\-M\-E/local/lib/libatlas.so}) before installing \-Scipy. \item {\ttfamily lxml} is a \-Python interface to the libxml2 \-X\-M\-L parser. \-After much ado, \-I decided to use {\ttfamily lxml} instead of the {\ttfamily xml} module in \-Python's standard library for several reasons ({\ttfamily xml} contains only limited support for \-X\-Path, scrambles the ordering of attributes in an element, etc.) \-The downside is that it can be harder to install. \-Installation instructions can be found on the \href{http://lxml.de/installation.html}{\tt lxml website} but summarized here. \-The packages {\ttfamily libxml2} and {\ttfamily libxslt} need to be installed first, and in that order. \-On \-Ubuntu, run {\ttfamily sudo apt-\/get install libxml2-\/dev libxslt1-\/dev}. \-To compile from source, run {\ttfamily ./configure -\/-\/prefix=\$\-H\-O\-M\-E/local -\/-\/with-\/python=\$\-H\-O\-M\-E/local}. \-Then run {\ttfamily make} followed by {\ttfamily make install}. \-Python itself needs to be compiled with {\ttfamily -\/-\/enable-\/shared} for this to work. \-Finally, download and unzip {\ttfamily lxml}, then run {\ttfamily  python setup.\-py install -\/-\/prefix=\$\-H\-O\-M\-E/local }.\end{DoxyItemize}
\hypertarget{installation_create_doc}{}\subsection{\-Create documentation}\label{installation_create_doc}
\-This documentation is created by \-Doxygen with the \-Doxypy plugin. \-To create new documentation or expand on what's here, follow the examples in the source code or visit the \-Doxygen home page.

\-To create this documentation from the source files, go to the {\ttfamily doc} directory in the distribution and run {\ttfamily  doxygen doxygen.\-cfg } to generate the \-H\-T\-M\-L documentation and \-La\-Te\-X source files. \-Run the {\ttfamily add-\/tabs.\-py} script to generate the extra navigation tabs for the \-H\-T\-M\-L documentation. \-Then go to the {\ttfamily latex} directory and type in {\ttfamily make} to build the \-P\-D\-F manual (\-You will need a \-La\-Te\-X distribution for this.) \-All of this is automated by running {\ttfamily make-\/all-\/documentation.\-sh}.\hypertarget{installation_install_gmxx2}{}\subsection{\-Installing G\-R\-O\-M\-A\-C\-S-\/\-X2}\label{installation_install_gmxx2}
{\itshape  \-G\-R\-O\-M\-A\-C\-S-\/\-X2 is not required for \-Force\-Balance and is currently deprecated. \-Installation is not recommended. \-This section is retained for your information and in case \-I choose to revive the software. \/}

\-I have provided a specialized version of \-G\-R\-O\-M\-A\-C\-S (dubbed version 4.\-0.\-7-\/\-X2) on the \href{https://simtk.org/home/forcebalance/}{\tt \-Sim\-T\-K website} which interfaces with \-Force\-Balance through the abinitio\-\_\-gmxx2 module. \-Although interfacing with unmodified simulation software is straightforward, \-G\-R\-O\-M\-A\-C\-S-\/\-X2 is optimized for force field optimization and makes things much faster.

\-G\-R\-O\-M\-A\-C\-S-\/\-X2 contains major modifications from \-G\-R\-O\-M\-A\-C\-S 4.\-0.\-7. \-Most importantly, it enables computation of the objective function and its analytic derivatives for rapid energy and force matching. \-There is also an implementation of the \-Q\-T\-P\-I\-E fluctuating-\/charge polarizable force field, and the beginnings of a \-G\-R\-O\-M\-A\-C\-S/\-Q-\/\-Chem interface (carefully implemented but not extensively tested). \-Most of the changes were added in several new source files (less than ten)\-: {\ttfamily qtpie.\-c}, {\ttfamily fortune.\-c}, {\ttfamily fortune\-\_\-utils.\-c}, {\ttfamily fortune\-\_\-vsite.\-c}, {\ttfamily fortune\-\_\-nb\-\_\-utils.\-c}, {\ttfamily zmatrix.\-c} and their corresponding header files, and {\ttfamily fortunerec.\-h} for the force matching data structure. \-The name 'fortune' derives from back when this code was called \-For\-Tune.

\-The force matching functions are turned on by calling {\ttfamily mdrun} with the command line argument {\ttfamily '-\/fortune'} ; without this option, there should be no impact on the performance of normal \-M\-D simulations.

\-Force\-Balance interfaces with \-G\-R\-O\-M\-A\-C\-S-\/\-X2 through the functions in {\ttfamily abinitio\-\_\-gmxx2.\-py} ; the objective function and derivatives are computed and printed to output files. \-The interface is defined in {\ttfamily fortune.\-c} on the \-G\-R\-O\-M\-A\-C\-S side. \-Force\-Balance needs to know where the \-G\-R\-O\-M\-A\-C\-S-\/\-X2 executables are located, and this is specified using the {\ttfamily gmxpath} option in the input file.\hypertarget{installation_install_gmxx2_prerequisites}{}\subsubsection{\-Prerequisites for G\-R\-O\-M\-A\-C\-S-\/\-X2}\label{installation_install_gmxx2_prerequisites}
\-G\-R\-O\-M\-A\-C\-S-\/\-X2 needs the base \-G\-R\-O\-M\-A\-C\-S requirements and several other libraries.

\begin{DoxyItemize}
\item \-F\-F\-T\-W version 3.\-3 \item \-G\-Lib version 2.\-0 \item \-Intel \-M\-K\-L library\end{DoxyItemize}
\-G\-Lib is the utility library provided by the \-G\-N\-O\-M\-E foundation (the folks who make the \-G\-N\-O\-M\-E desktop manager and \-G\-T\-K+ libraries). \-G\-R\-O\-M\-A\-C\-S-\/\-X2 requires \-G\-Lib for its hash table (dictionary).

\-G\-Lib and \-F\-F\-T\-W can be compiled from source, but it is much easier if you're using a \-Linux distribution with a package manager. \-If you're running \-Ubuntu or \-Debian, run {\ttfamily sudo apt-\/get install libglib2.\-0-\/dev libfftw3-\/dev}; if you're using \-Cent\-O\-S or some other distro with the yum package manager, run {\ttfamily sudo yum install glib2-\/devel.\-x86\-\_\-64 fftw3-\/devel.\-x86\-\_\-64} (or replace {\ttfamily x86\-\_\-64} with {\ttfamily i386} if you're not on a 64-\/bit system.

\-G\-R\-O\-M\-A\-C\-S-\/\-X2 requires the \-Intel \-Math \-Kernel \-Library (\-M\-K\-L) for linear algebra. \-In principle this requirement can be lifted if \-I rewrite the source code, but it's a lot of trouble, plus \-M\-K\-L is faster than other implementations of \-B\-L\-A\-S and \-L\-A\-P\-A\-C\-K.

\-The \-Intel \-M\-K\-L can be obtained from the \-Intel website, free of charge for noncommercial use. \-Currently \-G\-R\-O\-M\-A\-C\-S-\/\-X2 is built with \-M\-K\-L version 10.\-2, which ships with compiler version 11.\-1/072 ; this is not the newest version, but it can still be obtained from the \-Intel website after you register for a free account.

\-After installing these packages, extract the tarball that you downloaded from the website using the command\-:

\begin{DoxyVerb}tar xvjf gromacs-[version]-x2.tar.bz2 \end{DoxyVerb}


\-The directory structure is identical to \-G\-R\-O\-M\-A\-C\-S 4.\-0.\-7, but \-I added some shell scripts. {\ttfamily \-Build.\-sh} will run the configure script using some special options, compile the objects, create the executables and install them; you will probably need to modify it slightly for your environment. \-The comments in the script will help further with installation.

\-Don't forget to specify the install location of the \-G\-R\-O\-M\-A\-C\-S-\/\-X2 executables in the \-Force\-Balance input file! 