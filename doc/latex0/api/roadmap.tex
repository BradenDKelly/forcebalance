Force\-Balance is a work in progress and is continually being improved and expanded! Here are some current and future project development ideas.

Some notes on updating the version\-:

The formalism for the version number goes something like this\-:

v1.\-2.\-1\mbox{[}ab\mbox{]}\mbox{[}1-\/9\mbox{]}-\/123

where\-: \begin{DoxyItemize}
\item 1.\-2.\-1 stands for major release (may break compatibility), medium-\/sized release (important features), minor release (improvements) \item a or b, if present, stands for alpha or beta release, with numbers standing for the n-\/th alpha or beta release \item -\/123 stands for the number of commits that follow the manually specified release.\end{DoxyItemize}
To manually specify a release, create a tag and push it to the remote repository\-: git tag -\/a v1.\-2.\-1 -\/m \char`\"{}version 1.\-2.\-1\char`\"{} git push --tags

The version number should then be automatically generated by \char`\"{}git describe\char`\"{} which is run by setup.\-py at installation.

Finally, remember to update the version number in the documentation generation scripts!\hypertarget{roadmap_recent}{}\subsection{Most Recently Implemented\-:}\label{roadmap_recent}
\begin{DoxyItemize}
\item Implementing and expanding support for running target calculations in parallel across multiple nodes \item Development of a Force\-Balance G\-U\-I interface to complement the current command line interface \item Expand unit test cases to provide more thorough coverage of forcebalance modules\end{DoxyItemize}
\hypertarget{roadmap_current}{}\subsection{Current Development Goals, for version 1.\-2.\-2\-:}\label{roadmap_current}
\begin{DoxyItemize}
\item More comprehensive tutorial to walk users through the initial process of setting up targets and preparing for a successful Force\-Balance run\end{DoxyItemize}
\hypertarget{roadmap_longterm}{}\subsection{Longterm Development Ideas}\label{roadmap_longterm}
\begin{DoxyItemize}
\item Visualization of running calculations \end{DoxyItemize}
